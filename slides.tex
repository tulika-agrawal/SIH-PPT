%%%%%%%%%%%%%%%%%%%%%%%%%%%%%%%%%%%%%%%%%%%%%%%%%%%%%%%%%%%%%%%%%%%%%%%%%%%%%%%%%%%%%%%%%%%%%%
% Template Beamer Sugestivo para Projetos no Senac
% by ezefranca.com
% Based on MIT Beamer Template
% As cores laranja e azul seguem o padrao proposto no manual de uso da identidade visual senac
%%%%%%%%%%%%%%%%%%%%%%%%%%%%%%%%%%%%%%%%%%%%%%%%%%%%%%%%%%%%%%%%%%%%%%%%%%%%%%%%%%%%%%%%%%%%%% 

%\documentclass{beamer} %voce pode usar este modelo tambem
\documentclass[handout,t]{beamer}
\usepackage{graphicx,url}
\usepackage[brazil]{babel}   
\usepackage[utf8]{inputenc}
\usepackage{graphics}
\batchmode
% \usepackage{pgfpages}
% \pgfpagesuselayout{4 on 1}[letterpaper,landscape,border shrink=5mm]
\usepackage{amsmath,amssymb,enumerate,epsfig,bbm,calc,color,ifthen,capt-of}
\usetheme{Berlin}
\usecolortheme{senac}

%-------------------------Titulo/Autores/Orientador------------------------------------------------
\title[]{Idea/Approach Details}
\date{}
\author[]{Ministry/ Organization name: Ministry of External Affairs \\Problem Statement: App for top ten OWASP vulnerabilities scan \\Team Name: ABHYUDAYA\\Team Leader Name: Gaurav Garg\\ College Code: U-0013}

%-------------------------Logo na parte de baixo do slide------------------------------------------
\pgfdeclareimage[height=0.9cm, keepaspectratio]{senac-logo}{logo.pdf}
\logo{\pgfuseimage{senac-logo}\hspace*{0.5cm}}

%-------------------------Este código faz o menuzinho bacana na parte superior do slide------------
\AtBeginSection[]
{
  \begin{frame}<beamer>
    \frametitle{Outline}
    \tableofcontents[currentsection]
  \end{frame}
}
\beamerdefaultoverlayspecification{<+->}
% -----------------------------------------------------------------------------
\begin{document}
% -----------------------------------------------------------------------------

%---Gerador de Sumário---------------------------------------------------------
\frame{\titlepage}
\section[]{}
% \begin{frame}{Sumário}
%   \tableofcontents
% \end{frame}
%---Fim do Sumário------------------------------------------------------------


% -----------------------------------------------------------------------------
\section{Idea / Approach details}
\begin{frame}{Idea/Approach details}
%introducao
Describe your idea / Solution / Prototype here
\\
Describe your Technology stack here
\end{frame}
%------------------------------------------------------------------------------

%------------------------------------------------------------------------------
\section{Idea / Approach details}
\begin{frame}{Idea / Approach details}
%referencial teorico, estado da arte, etc
Describe your Use Cases here
Describe your Dependencies / Show stopper here
\\
(crucial data or software or hardware, without which you cannot create the final
model/ product)
\end{frame}
%------------------------------------------------------------------------------

%------------------------------------------------------------------------------
%------------------------------------------------------------------------------

% -----------------------------------------------------------------------------
\end{document}
%-----------------------------------------------Este comentario nunca aparecera